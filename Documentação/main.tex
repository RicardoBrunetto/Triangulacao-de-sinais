% Pacotes
\documentclass[12pt]{article}

\usepackage{adjustbox}
\usepackage[utf8]{inputenc}
\usepackage{amsmath}
\usepackage{sbc-template}
\usepackage{amsmath}
\usepackage{fancyvrb}
\usepackage{relsize}
\usepackage{graphicx,url}

\newcommand{\pder}[2]{\frac{\partial#1}{\partial#2}}


\sloppy

\title{Estimação de Localização com Triangulação de Sinais\\Baseado na Potência da Fonte Emissora}

%TODO: Inserir nome
\author{Ricardo Henrique Brunetto\inst{1}}


\address{Departamento de Informática -- Universidade Estadual de Maringá (UEM)\\
	Maringá -- PR -- Brasil
	%TODO: Inserir e-mail
	\email{ra94182@uem.br}
}

\begin{document}

	\maketitle
	%
	% \begin{abstract}
	% 	This paper makes use of known trigonometric relations in order to develop a
	% 	computational strategy that allows to calculate the sine of any angles of the
	% 	trigonometric circle from the first six terms of the Taylor Series that defines it,
	% 	so that	the result has ten decimal digits of precision in the minor possible computational
	% 	time.
	% \end{abstract}

	\begin{resumo}
		Este artigo faz uso de métodos algébricos e numéricos para apresentar uma solução computacional
		no problema da triangulação de sinais, usando como fator de cálculo a potência da fonte emissora,
		com uma precisão específica no menor tempo computacional possível. Além disso, são discutidos, neste artigo,
		fatores que podem levar à imprecisão das estimativas.
	\end{resumo}

  \section{Introdução}
	\label{sec:introducao}
	%TODO: Intrudução
	%É de grande importância ser capaz de estimar...
	Embora existam diversas formas de analisar as características dos dados de sinais para o cômputo
	da estimativa (natureza do sinal, intervalos de tempo entre envio e recebimento, ângulo do sinal, etc),
	será utilizada a potência do sinal. Isso porque existem modelos matemáticos mais bem
	estruturados e precisos, que são capazes de embasar com maior fidelidade a estimativa.

	\section{Metodologia}
	\label{sec:metodologia}
	%TODO: Explicar como a bagaça será feita
	No modelo adotado para experimentação, elaborado por \cite{polidorio:triangulacao}, a fonte emissora é disposta
	em um sistema de coordenadas de três dimensões onde $m$ receptores são arranjados em posições $(x_k, y_k, z_k)$, onde
	cada coordenada indica a distância (em \underline{metros}) do receptor $k$ até a origem seguindo o respectivo eixo. Isso implica que as coordenadas
	são dadas em metros, o que servirá de apoio para desenvolvimento dos cálculos.
	Os casos experimentais de \cite{polidorio:triangulacao} utilizam $5$ receptores e serão abordados posteriormente.

	\section{Fundamentação Teórica}
	\label{sec:fund_teor}
	%TODO: Expor como serão estimadas as distâncias / Abordar os métodos
	%Como um receptor se comporta

	\paragraph{Estimativa da Distância em Função da Potência do Sinal:}
	A potência do sinal (em dBm) da fonte emissora serve como parâmetro para a estimativa da distância.
	Tal estimativa é baseada na \textbf{potência de referência} ($p^k_0$), medida entre o emissor e o receptor $k$ a $1$ metro de distância,
	\textbf{fator de atenuação} do sinal do emissor até o receptor $k$ ($\mathcal{L}^k$). Assim, pode-se aferir, com certa precisão, a \textbf{distância} da fonte emissora em relação ao receptor $k$
	($d_k$) baseada na \textbf{potência do sinal} registrado pelo receptor $k$ ($p_k$) através da seguinte equação:
	\begin{align}
		\mathlarger{d_k = 10^{\frac{p^k_0 - p^k}{10\mathcal{L}^k}}} \label{eq:distancia}
	\end{align}

	\paragraph{Método de Newton-Raphson:}
	\begin{align}
			x_{i+1} = x_i - \frac{f(x_i)}{f'(x_i)} \label{eq:nr}
	\end{align}

	\section{Desenvolvimento}
	\label{sec:dev}
	%TODO: Apresentar as equações com base nos casos testes e o algoritmo
	Esta seção será particionada em duas outras seções: \textbf{Desenvolvimento Matemático}, onde
	se deseja construir a solução matemática para o problema da estimativa da triangulação de sinais;
	\textbf{Desenvolvimento Computacional}, onde se deseja implementar os princípios e a lógica da
	solução desenvolvidos matemáticamente para a construção de uma solução computacional que seja
	o mais rápida possível para determinada precisão, levando em conta demais aspectos e
	limitações computacionais.

	\subsection{Desenvolvimento Matemático}
	\label{sec:dev_math}
	Supõe-se, portanto, $m$ receptores dispostos em um sistema de coordenadas tridimensional, conforme
	exposto na Seção \ref{sec:metodologia}. Para cada sinal que a fonte emissora disparar, os $m$
	receptores o receberão e, com base em sua potência, calcularão, individualmente, a distância
	estimada que cada qual está do emissor ($d_k$, conforme a equação \ref{eq:distancia}).

	Conforme já mencionado, cada receptor $k$ é capaz de cobrir uma região esférica.
	Assim, cada receptor $k$ contribuirá com uma equação da forma
	\begin{align}
		f_k(x,y,z) = (x - x_k)^2 + (y - y_k)^2 + (z - z_k)^2 = d_k \label{eq:receptor}
	\end{align}
	onde
	\begin{itemize}
		\item $x_k, y_k, z_k$ são as coordenadas do receptor $k$ (conhecidas);
		\item $x, y, z$ são as coordenadas da fonte emissora (descconhecidas);
		\item $d_k$ é a distância estimada do receptor à fonte (conhecida).
	\end{itemize}

	Serão, portanto, $m$ equações não-lineares da forma acima. Logo, ter-se-á:

	\begin{align*}
		F(x,y,z) =
		\left \{
		\begin{array}{cl}
			(x - x_1)^2 + (y - y_1)^2 + (z - z_1)^2 = d_1\\
			(x - x_2)^2 + (y - y_2)^2 + (z - z_2)^2 = d_2\\
			\vdots \\
			(x - x_m)^2 + (y - y_m)^2 + (z - z_m)^2 = d_m\\
		\end{array}
		\right.
	\end{align*}
	$$\iff$$
	\begin{align*}
		F(x,y,z) =
		\left \{
		\begin{array}{cl}
			x^2 - 2x_1x + x_1^2 + y^2 - 2y_1y + y_1^2 + z^2 - 2z_1z + z_1^2 = d_1\\
			x^2 - 2x_2x + x_2^2 + y^2 - 2y_2y + y_2^2 + z^2 - 2z_2z + z_2^2 = d_2\\
			\vdots \\
			x^2 - 2x_mx + x_m^2 + y^2 - 2y_my + y_m^2 + z^2 - 2z_mz + z_m^2 = d_m\\
		\end{array}
		\right.
	\end{align*}

	Deseja-se, portanto, obter as raízes de $F$, pois $F(x,y,z) = 0$ implica que a posição
	da fonte emissora, de acordo com todos os $m$ receptores é melhor estimada em $(x,y,z)$.
	Para tanto, deve-se linearizar o sistema. Dessa forma,
	\[
	\begin{array}{cl}
		x^2 - 2x_1x + x_1^2 + y^2 - 2y_1y + y_1^2 + z^2 - 2z_1z + z_1^2 = d_1\\
		x^2 - 2x_2x + x_2^2 + y^2 - 2y_2y + y_2^2 + z^2 - 2z_2z + z_2^2 = d_2\\
		\vdots \\
		x^2 - 2x_mx + x_m^2 + y^2 - 2y_my + y_m^2 + z^2 - 2z_mz + z_m^2 = d_m\\
	\end{array}
	\]

	Conforme explanado na Seção \ref{sec:fund_teor}, o Método dos Mínimos Quadrados pode ser aplicado
	para normalizar sistemas de equações, removendo ambiguidades sem descartar equações.

	\subsection{Desenvolvimento Computacional}
	\label{sec:dev_comp}
	Esta seção será particionada nos três pontos necessários para a construção da desejada solução
	computacional: \textbf{Implementação}, onde a solução matemática exposta em \ref{sec:dev_math}
	será codificada, bem como as estruturas de dados necessárias, a fim de garantir uma determinada
	precisão; \textbf{Otimização}, onde serão abordados aspectos referentes ao tempo computacional
	e formas de minimizá-lo; \textbf{Limitações}, onde serão expostas as limitações da solução e suas
	possíveis falhas.
	\subsubsection{Implementação}
	\label{sec:imple}
	\subsubsection{Otimização}
	\label{sec:opt}
	\subsubsection{Limitações}
	\label{sec:lim}

	\section{Discussão}
	\label{sec:discussao}
	%TODO: Debater a respeio da eficiência do algoritmo, imprecisão e erros nos dados e como consertar

	\section{Conclusão}
	\label{sec:conclusao}
	%TODO: Expor a análise a respeito do problema e como é solucionado pelo que foi proposto.

	%TODO: Corrigir as referências
	\bibliographystyle{sbc}
	\bibliography{references}
\end{document}
